\documentclass{vldb}
\usepackage{graphicx}
\usepackage{balance} 
\usepackage{url}
\usepackage{graphicx}
\usepackage{xcolor}
\usepackage{bera}
\usepackage{listings}

\iffalse
  From stackexchange http://tex.stackexchange.com/questions/83085/how-to-improve-listings-display-of-json-files
\fi

\colorlet{punct}{red!60!black}
\definecolor{background}{HTML}{EEEEEE}
\definecolor{delim}{RGB}{20,105,176}
\colorlet{numb}{magenta!60!black}
\lstdefinelanguage{json}{
    basicstyle=\normalfont\ttfamily,
    numbers=left,
    numberstyle=\scriptsize,
    stepnumber=1,
    numbersep=8pt,
    showstringspaces=false,
    breaklines=true,
    frame=lines,
    backgroundcolor=\color{background},
    literate=
     *{0}{{{\color{numb}0}}}{1}
      {1}{{{\color{numb}1}}}{1}
      {2}{{{\color{numb}2}}}{1}
      {3}{{{\color{numb}3}}}{1}
      {4}{{{\color{numb}4}}}{1}
      {5}{{{\color{numb}5}}}{1}
      {6}{{{\color{numb}6}}}{1}
      {7}{{{\color{numb}7}}}{1}
      {8}{{{\color{numb}8}}}{1}
      {9}{{{\color{numb}9}}}{1}
      {:}{{{\color{punct}{:}}}}{1}
      {,}{{{\color{punct}{,}}}}{1}
      {\{}{{{\color{delim}{\{}}}}{1}
      {\}}{{{\color{delim}{\}}}}}{1}
      {[}{{{\color{delim}{[}}}}{1}
      {]}{{{\color{delim}{]}}}}{1},
}

\begin{document}

\title{Providing better confidentiality and authentication on the Internet using Namecoin and MinimaLT}

\numberofauthors{1}
\author{
\alignauthor
Frederic Jacobs\\
\affaddr{www.fredericjacobs.com}\\
\email{me@fredericjacobs.com}
}

\maketitle

\begin{abstract}
In this paper, we introduce a duo of improvements for the Internet that would lead to better security. The authentication model on the Internet is broken and TLS connections have a considerable overhead. We try to address those issues with changes at both the application layer, introducing a replacement for the DNS system, and at the transport layer, a drop-in replacement for TCP built on top of UDP that requires no changes at the network layer.
\end{abstract}

\section{Introduction}

\subsection{Defining user privacy}
The solutions brought forward in this paper are attempts to fix confidentiality and authentication on the Internet. Anonymity is not provided. An attacker could still get a significant amount of metadata. Unfortunately, because MinimaLT runs over UDP you can't connect through the Tor network.\footnote{If MinimaLT proves to be a safer and faster alternative to TLS, I imagine that the Tor project would look into implementing it to speed up the network and make relay connections safer.}
\subsection{Motivation}

When the Internet was designed at DARPA, the primary goal was to design a system that could provide interconnection between multiple computers. The Web then came by with the motivation to be able to freely exchange information. The Internet has mainly been used for open communications. Any computer on the network could request files. But, over time, people started trusting the internet more and more and with the appearance of services. But the Internet grew so quickly out of what DARPA proposed for a trusted environment. The Internet was never designed to be ran by so many different entities and the threat model implied that you had to trust the entities running the infrastructure of the internet.

\section{Domain names and authenticity}

In today's model, if I want to load a page from \emph{facebook.com}, my computer will have to first get the Domain Name System records matching that domain. DNS was designed in a hierarchical way and TLD registrations are handled by a single organisation, the ICANN.

So what is wrong with DNS?

When the original Domain Name System was designed, it did not include security; instead it was designed to be a scalable distributed system. The DNSSEC attempted to add security, while maintaining backwards compatibility. Those security extensions added public key signing to DNS zones but who is signing those zones? The DNS Root Zone, of course. You will then have to trust those too. We thus consider DNSSec as an attempt to try to fix a broken system. We want to design a distributed system where anyone can register a domain but without having a central registration authority.

But how can we verify authenticity? 
Even if my domain name system returns the right IP address how do I know for sure that I'm establishing a connection with the client I want. Today, we are using another hierarchical system to verify authenticity, namely SSL certificates. This means that in addition to trusting ICANN, we will have to trust hundreds of Root Certificate Authorities that are shipped with out browsers.\cite{mozillaSSL}

If only one of those 100s of CA gets compromised, it could result in the man-in-the-middling of any user without any warning since a root certification authority can generate a fake valid certificate for any website. This is what happened in Iran after the Dutch certificate authority DigiNotar got compromised\cite{diginotarHack}.

Now that we are convinced that the hierarchical trust model of the internet is broken, what measures have already been taken to fix authentication on the Internet?

\subsubsection{Certificate Pinning}

The Chrome Security team, led by Adam Langley led the way by implementing certificate pinning. Certificate pinning is a reasonably effective measure for a centralised internet, where only a few websites gather most of the traffic. Certificate pinning works by shipping \emph{pins} in the browser's binary.\cite{chromiumPins} Every time a user loads a pinned website, the certificate fingerprint is compared to the one provided in the binary. If the signature matches, the client continues the SSL handshake. Otherwise, an error message is shown to the user explaining a secured connection couldn't be established.

Although this is a very efficient method to verify SSL certificates, this method doesn't scale and is hard to maintain at a larger scale. In addition to that, the Chromium team needs to verify the "pin" definition before merging every pin request into the code branch. Therefore only larger websites do have certificate pins.

\subsubsection{TACK}

TACK is a proposal by Moxie Marlinkspike and Trevor Perrin, it's a way to 'pin' TLS servers to the correct public key even when a Certificate Authority is telling you differently. Although this is a promising proposition, it doesn't protect against an attacker that has a long-term MITM capability since pins are set on the first connections and do expire after some time.\cite{tackMITM}

\subsubsection{DANE}

The IETF proposal called \emph{DANE} is an attempt at making large scale certificate pinning but by distributing the certificate fingerprint by DNS. This would enable website owners to specify their certificate fingerprint as a DNS entry and visitors would thus be able to verify the authenticity of the server.

Even if we consider that DNSSec does provide good security, this system does still rely on the trust on ICANN and will thus not match the security we want to achieve.

\subsubsection{Tor Hidden Services}

Tor hidden services are reachable by hashes of public keys. This is of course the ideal case when it comes to security because the address contains itself information about the key itself. Unfortunately, humans are not good at remembering pseudorandom 16-character long strings.

\subsection{Zooko's triangle}
In this section, we are introducing Zooko's triangle conjecture and our attempt to square it.

\begin{figure}[h!]
\centering
\includegraphics[width=0.3\textwidth]{ZookoTriangle.png}
\end{figure}

Zooko's triangle says that out of these three properties \cite{zookoTriangleWikipedia}, you can usually take only two.
\begin{itemize}
\item \emph{Secure}\footnote{We can't agree with the naming of this property given the threat model we described previously in this paper.}: The quality that there is one, unique and specific entity to which the name maps.
\item \emph{Global}: The lack of a centralized authority for determining the meaning of a name. Instead, measures such as a Web of trust are used.
\item \emph{Memorable}: The quality of meaningfulness and memorability to the users of the naming system.
\end{itemize}

We can thus see that the systems proposed so far only gather two out of three of those properties. If we take the DNS(Sec) system with DANE extensions, we can have a memorable address that is "secure" but unfortunately doesn't have the global property because the ICANN is a centralized authority. Alternatively, Tor's Onion addresses do have the "secure" property and are global but a 16 character pseudorandom string is not memorable.

\subsection{Squaring Zooko's Triangle}

In the following section we are introducing a naming system that is an attempt at squaring Zooko's triangle.

Back in January 2011, Aaron Swartz described on his blog about how Bitcoin blockchain could help in squaring Zooko's triangle. A few months later, a first implementation of that idea came into existence, Namecoin.

\subsubsection{The Bitcoin Blockchain}

The blockchain is Bitcoin's main innovation. Blockchains are mainly linear data-structures that were invented specifically for the Bitcoin project to store the history of all past transactions but they can be applied anywhere a distributed consensus needs to be established in the presence of malicious or untrustworthy actors.

To understand how they work we will cover the basics of how Bitcoin works. Let's take an example and see what happens when Alice tries to transfer money to Bob. Every user on the network has one address. Bitcoin addresses are generated based a public key. 
\begin{center}
Key-Hash = RIPEMD-160(SHA-256(public key))
$\text{BTC}_{\text{Address}}$ = Base58(Version +\footnote{The + sign is a string concatenation} Key-Hash + Checksum)
\end{center}

Private-public keypairs are generated when making new Bitcoin addresses. The curved used in Bitcoin is secp256k1 which is surprisingly a NIST recommended curve.\cite{VOID}

Alice must know Bob's address to send him money. Now that Alice has Bob's address, she creates a new message saying she sends a few Bitcoins to Bob and uses her private key to generate an ECDSA signature. Once she has generated that message and signed it, she starts gossiping about her transaction on the network, her peers hear about the transaction, they verify if Alice has enough money to make the transaction and verify the signature, if the transaction looks legitimate, they start telling all of their peers. The verification can be done thanks to the blockchain datastructure which is a decentralized and unique record of all the transactions. Peers that are miners, eventually hear about this transaction and it's added to the transactions memory pool. This pool is a queue of transactions that are not yet merged in the blockchain. But now how can we merge transactions into the blockchain? 

\subsubsection{Proof of work}

The concept of \emph{proof of work} is used to merge the blockchain. It makes adding entries in the blockchain an expensive process computationally wise. Let's say Alice wants to send Bitcoins to Bob. Alice will start gossiping on the network, telling all her peers that she wants to send money to Bob. Every client, has a copy of the blockchain and can thus assess if Alice has the amount of money she wants to transfer to Bob. If she has so, gossip will spread.

Once the miners, the workers of the blockchain, learn about a  valid transaction (Alice has enough money to make the transaction and her signature is correct), they will add it to their memory pool. If the transaction is valid, the miners will add this transaction in the next block they will be mining. The benefit of making it costly to validate transactions is that validation can no longer be influenced by the number of network identities someone controls, but only by the total computational power they can bring to bear on validation.

So what is mining technically?

The hard challenge that is used in Bitcoin that needs to be solved is based on the strength of cryptographic hashes, also known as one-way functions. We consider that it is hard for someone to come up with the parameters of a hash functions for a given result. The function used in Bitcoin is \emph{SHA-256} but this hash function could be substituted by any other. Another cryptocurrency, Litecoin, chose to use the \emph{scrypt} function.

If I want to add some blocks (list of transactions) in the blockchain, I will have to solve this problem

SHA-256("TransactionsInfo" + challengeNumber) =< target

where \emph{transactionsInfo} is a parameter list of information about the transactions in the blocks (and some extra information like a return address for the reward). The blockchain is vulnerable to some malleability regarding some informations in the transaction but all important information (such as the amount of transaction, recipient and sender) is part of this hash.
The challenge the miner has to solve – the proof-of-work – is to find the challenge number such that when we append the transactions infos to the challenge number and hash the combination the output hash is smaller than a certain number.
We notice that this certain challenge number is established by the network and determines how hard the problem is. In Bitcoin, this number is dynamically adjusted to keep an approximate block validation time of 10 minutes.\cite{hashCash}

hash("Hello Blockchain) examples

When someone succeeds in solving this problem, he sends his solution to the network. Nodes verify if that answer is valid, and if it is, they broadcast it to their peers. And so it progressively spreads across all nodes and is added to the blockchain. 

\subsubsection{Dealing with collisions}

Now what happens if two nodes, from separate parts of the blockchain do succeed in solving the challenge at almost the same time. Both nodes and their peers will spread different versions of the blockchain. We say that the blockchain has \emph{forked}. How do we solve this?

In this case, miners will start mining the next block based on the version of the blockchain they have. If they hear that another blockchain is longer than the one they were working on before, they will switch to the longer one and put the transactions in the orphan blocks (blocks that were in the previous fork) back into the memory pool if they were not merged. 

We can now understand that because every node chooses to have the longest blockchain possible, it will be very hard for an attacker to spread a fake version of the blockchain because this would involve solving the challenge for every preceding block because blocks are chained and must contain the block identifier of the previous one. 

Why would one mine and spend so much computational power?

Miners are rewarded for their efforts. First, when doing a transaction, I can speed up the transfer of the money buy adding a transaction fee that will go directly to the miners. Mining software is thus optimised to sort the transactions to be merged in blocks in decreasing order of transaction fee. 
The other reward from mining comes from the coinbase transaction, mining does generate money.  At the creation of Bitcoin, this reward was set to be a 50 BTC. But for every 210,000 validated blocks (once every four years) the reward halves. This has happened just once, to date, and so the current reward for mining a block is 25 bitcoins. This halving in the rate will continue every four years until the year 2140 CE. At that point, the reward for mining will drop below $10^{-8}$ bitcoins per block which is a satoshi, the smallest unit of Bitcoin and the total amount of bitcoins will cease to increase.

\subsubsection{From Bitcoin to Namecoin}

Now that we understand how blockchains work and why they are safe data-structures, let's now see how we can use them to square Zooko's triangle. 

Namecoin is a bitcoin fork that was designed as a decentralized key-value store.

Putting information in the blockchain does cost a certain price hence

Let's say Alice wants to register a domain name name. To achieve that, Alice needs money, Namecoins.  

Just like Bitcoin, Namecoin is a crypto-currency but in addition to being a cryptocurrency, Namecoin doubles as being a decentralized key-value store.

Here are a few other possible applications of a distributed key-value store:
\begin{itemize}
\item Aliases: The blockchain can be used to store an easy to remember alias for a GPG/SSH key, a Bitcoin address or any other cryptographic identity.
\item Timestamping: The blockchain could store information about a specific file and from a hash of that we could find matching author name, owner, etc.
\item Messaging: The blockchain could be a decentralized store for  long-term messages vs BitMessage.
\end{itemize}

Writing data in the blockchain does have a certain price. Registering a domain does cost the registration fee (0.01NMC that goes to nobody) plus the transaction fee (that goes to the miner who succeeds in adding the block that contains this transaction).

The cost includes a network fee and a transaction fee. The fees are denominated in Namecoins (\emph{NC}). Initially, the network fee was 50 NC but it decreases twice every 2 months, which means that it is already less than 1 NC after a year. This design was meant to make it expensive to register domains in the first few months to avoid the issue of domain name squatting.

Let's see what a domain name value message looks like to understand how it squares Zooko's triangle.

\begin{lstlisting}[language=json,firstnumber=1]
{
    "ip"      : "209.236.123.133",
    "ip6"     : "IPV6",
    "tor"     : "Tor Hidden Service.onion",
    "email"   : "me@fredericjacobs.com",
    "info"    : "Frederic Jacobs",
    //"service" : [ ["smtp", "tcp", 10, 0, 25, "mail"] ]
    "tls": {
        "tcp": {
            443: [[1, "30F38EDAABC67F0344DBE27018552F7D575946EF", 1]]
        }
    }
    "map":
    {
        "www" : { "ip": "209.236.123.133" },
    }
}
\end{lstlisting}

A namecoin domain needs to be renewed every 36,000 blocks which at the current rate is around 200 days. Those updates are free. Hence, unlike ICANN domain names, you don't have to pay renewal fees.

\subsection{Known Issues with this new model}

To be able to resolve Namecoin domains, you need to have a local copy  of the complete up-to-date blockchain on your machine. In a mobile world this is currently not acceptable due to performance, bandwidth and storage restrictions. An alternative could be to deploy DNSNMC servers which would work plug-and-play with the current DNS server clients. A few of them have already been deployed. The issue being that you will have to trust a DNSNMC server.

Another issue that still needs to be addressed is domain squatting. Because registering namecoin domains (.bit) became ridiculously cheap, a lot of domains are being squatted by people hoping to resell those domains at some point in the future. A better pricing system that prevents massive domain registration should be adopted because registration costs did decrease too quickly to be an effective counter-measure.

\section{Transport security}
What's wrong with tcp is slow and insecure. How to move away from it? Well, we don't really have any other option to base it on UDP.
But we love reliability!

Minimalt  

Multipath TCP advantage?

Doesn't solve anonymity ==> Tor

Issues with too big frames for firewalls?
\section{The {\secit Body} of The Paper}


\section{Conclusions}

% ensure same length columns on last page (might need two sub-sequent latex runs)
//\balance

%ACKNOWLEDGMENTS are optional
\section{Acknowledgments}
Most of the Namecoin research is based on Greg Slepak's work.
\bibliographystyle{abbrv}
\bibliography{bibliography}
\subsection{References}

\begin{appendix}
Define Perfect forward secrecy

Using Elliptic curve crypto but not backdoored.

Impact on censorship( when encrypted blobs can be transferred)
\section{Final Thoughts}
This paper brought forward Namecoin as a solution that squares Zooko's triange and thus provide good addressing.
\end{appendix}

\end{document}
